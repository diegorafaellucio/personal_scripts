\documentclass[12pt,a4paper]{article}
\usepackage[brazil]{babel}
\usepackage[utf8]{inputenc}
\usepackage{hyperref}
\usepackage{enumitem}
\usepackage{graphicx}
\title{Manual Simples: Como Usar o Script de Predição OBB}
\author{}
\date{\today}
\begin{document}
\maketitle

\section*{O que este programa faz?}
Este programa permite que você use um modelo de Inteligência Artificial para analisar um vídeo e marcar automaticamente objetos detectados em cada quadro (imagem) do vídeo. O resultado são imagens com "caixinhas coloridas" desenhadas sobre os objetos identificados.

\section*{O que você precisa?}
\begin{itemize}
    \item Um vídeo em formato comum (ex: MP4).
    \item Um arquivo de modelo de IA (fornecido pela equipe, termina com .pt).
    \item Um computador com Windows ou Linux.
    \item \textbf{Apenas conhecimentos básicos de computador!}
\end{itemize}

\section*{Passo 1: Preparando o Computador}
\begin{enumerate}[label=\arabic*.]
    \item \textbf{Abra o terminal (Linux) ou o Prompt de Comando (Windows).}
    \item Navegue até a pasta do programa:
    \begin{verbatim}
cd /home/diego/Projects/PERSONAL/scripts/modules/predict_data_model
\end{verbatim}
    (No Windows, adapte o caminho conforme sua pasta)
    \item Crie um ambiente virtual (opcional, mas recomendado):
    \begin{verbatim}
python -m venv venv
\end{verbatim}
    \item Ative o ambiente virtual:
    \begin{itemize}
        \item Linux:
        \begin{verbatim}
source venv/bin/activate
\end{verbatim}
        \item Windows:
        \begin{verbatim}
venv\Scripts\activate
\end{verbatim}
    \end{itemize}
    \item Instale os programas necessários:
    \begin{verbatim}
pip install -r requirements.txt
\end{verbatim}
\end{enumerate}

\section*{Passo 2: Rodando o Programa}
\begin{enumerate}[label=\arabic*.]
    \item Coloque o arquivo do vídeo e o arquivo do modelo na mesma pasta (ou saiba o caminho deles).
    \item No terminal, execute o comando abaixo, trocando os caminhos conforme seus arquivos:
    \begin{verbatim}
python predict_data_model.py --video SEU_VIDEO.mp4 --model SEU_MODELO.pt --output pasta_resultado
\end{verbatim}
    \item Após rodar, as imagens com marcações serão salvas na pasta que você escolheu em --output.
\end{enumerate}

\section*{Dicas e Problemas Comuns}
\begin{itemize}
    \item Se nada for detectado nas imagens, confirme se o modelo e o vídeo são compatíveis.
    \item Se não souber como "navegar até a pasta" ou "ativar o ambiente virtual", peça ajuda!
\end{itemize}


\end{document}
