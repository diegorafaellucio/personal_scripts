%% start of file `template.tex'.
%% Copyright 2006-2013 Xavier Danaux (xdanaux@gmail.com).

\documentclass[12pt,a4paper,sans]{moderncv}

% modern themes
\moderncvstyle{banking}
\moderncvcolor{black}

% character encoding
\usepackage[utf8]{inputenc}
\usepackage[brazil]{babel}

% adjust the page margins
\usepackage[scale=0.75]{geometry}

% personal data
\name{Diego Rafael}{Lucio}
\address{Maringá - Paraná - Brasil}{}{}
\phone[mobile]{+55(44)99950-0352 }
\email{diegorafaellucio@gmail.com}
\homepage{www.linkedin.com/in/diegolucio/}

\renewcommand*{\cventry}[7][.25em]{%
  \cvitem[#1]{}{%
    {\bfseries#3}\\%
    \ifthenelse{\equal{#4}{}}{}{{#4}}%
    \ifthenelse{\equal{#5}{}}{}{, {#5}}\\%
    \ifthenelse{\equal{#2}{}}{}{{#2}\vspace{2pt}}%
    \strut%
    \ifx&#7&%
    \else{\newline{}\begin{minipage}[t]{\linewidth}\small#7\end{minipage}}\fi\vspace{6pt}}
    }

\patchcmd{\makeheaddetailssymbol}
  {\textbullet}{}{}{}

\begin{document}

\makecvtitle

\vspace{12pt}
\section{RESUMO PROFISSIONAL}

\small{Doutor em Ciência da Computação pela Universidade Federal do Paraná, com mais de 10 anos de experiência combinando desenvolvimento full-stack e pesquisa avançada em Inteligência Artificial. Desenvolvedor full-stack sênior com sólida experiência em tecnologias modernas de backend (Python com Django/FastAPI e Java com Spring Boot) e frontend (React.js e Next.js). Especialista em Machine Learning e Visão Computacional, com histórico comprovado no desenvolvimento de soluções inovadoras em biometria e reconhecimento facial.}

\vspace{12pt}
\section{Competências}

\begin{itemize}
\item \textbf{Desenvolvimento Backend:}
  \begin{itemize}
    \item \textbf{APIs e Microserviços:} FastAPI, Django REST Framework, Spring Boot, Flask
    \item \textbf{Bancos de Dados:} MySQL, PostgreSQL, SQLServer, MongoDB, Redis, Neo4j
    \item \textbf{Cache e Performance:} Redis (cache, gerenciamento de sessão, pub/sub), Memcached
    \item \textbf{Bancos de Grafos:} Neo4j (modelagem de grafos, consultas Cypher)
    \item \textbf{Filas de Mensagens:} Apache Kafka, RabbitMQ, Redis Streams
  \end{itemize}
\vspace{2pt}

\item \textbf{Python e Ecossistema:}
  \begin{itemize}
    \item \textbf{Machine Learning:} PyTorch, TensorFlow, Scikit-learn, Keras, PyCaret
    \item \textbf{APIs e Web:} FastAPI, Django, Sanic, Flask
    \item \textbf{Análise de Dados:} Pandas, NumPy, Matplotlib, Seaborn
    \item \textbf{Processamento de Imagens:} OpenCV, PIL, scikit-image, imutils
  \end{itemize}
\vspace{2pt}

\item \textbf{Outras Linguagens:} Java (Spring, JSF, Hibernate), JavaScript/TypeScript, PHP, MATLAB, C++, C
\vspace{2pt}
\item \textbf{Desenvolvimento Web:} React, Angular, Spring Boot
\vspace{2pt}
\item \textbf{DevOps:} Docker, Kubernetes, Jenkins, CI/CD, AWS, Linux
\vspace{2pt}
\item \textbf{Metodologias:} Ágil (Scrum, Kanban, TDD), DevOps, Integração Contínua
\vspace{2pt}
\item \textbf{Outras Ferramentas:} Git, Jira, Confluence, Microsoft Azure
\end{itemize}

\vspace{12pt}
\section{Experiência Profissional}

\vspace{3pt}

\vspace{24pt}
\small{\textbf{Tokenology - Miami, EUA}}
\vspace{3pt}

\textbf{Engenheiro Sênior de Machine Learning / Desenvolvedor Back-End} \hfill \textit{Janeiro 2024 -- Presente}
\begin{itemize}
    \item Liderando o desenvolvimento de sistemas avançados de reconhecimento facial e detecção de vivacidade usando arquiteturas de vanguarda como Vision Transformers e modelos de atenção.
    \item Implementando técnicas inovadoras de detecção de fraudes em detecção de vivacidade, incluindo detecção de deepfake e ataques de apresentação.
    \item Desenvolvendo pipelines robustos de verificação biométrica facial com alta precisão e baixa latência usando PyTorch e ONNX Runtime.
    \item Coordenando a integração de sistemas biométricos com microserviços Spring Boot, garantindo escalabilidade e segurança.
    \item Implantei soluções de cache distribuído e gerenciamento de sessão, reduzindo tempos de resposta da API em 40\% e melhorando a escalabilidade do sistema para usuários concorrentes.
    \item \textbf{Tecnologias principais:} Python (PyTorch, ONNX, FastAPI), Java (Spring Boot), Docker, AWS, Redis
\end{itemize}

\vspace{12pt}
\textbf{Consultor de Machine Learning} \hfill \textit{Janeiro 2023 -- Dezembro 2023}
\begin{itemize}
    \item Desenvolvi modelos de reconhecimento facial usando arquiteturas modernas como ArcFace e CosFace, alcançando alta precisão em cenários desafiadores.
    \item Implementei sistema de detecção de vivacidade usando técnicas avançadas de deep learning para detecção de movimento e expressão facial.
    \item Criei APIs FastAPI escaláveis para processamento assíncrono de verificação biométrica.
    \item Otimizei modelos para inferência em tempo real usando ONNX Runtime e TensorRT.
    \item Implementei sistema de mensageria pub/sub para notificações em tempo real e camada de cache para predições de modelos frequentemente acessadas, melhorando performance do sistema em 35\%.
    \item \textbf{Tecnologias principais:} Python (PyTorch, TensorRT, FastAPI), Java (Spring Boot), Docker, AWS
\end{itemize}

\vspace{24pt}
\small{\textbf{Ecotrace - Vinhedo, Brasil}}
\vspace{3pt}

\textbf{Consultor Sênior de Inteligência Artificial} \hfill \textit{Janeiro 2024 -- Presente}
\begin{itemize}
    \item Liderando projetos de expansão e otimização do sistema de visão computacional para novas unidades industriais.
    \item Atuando como mentor técnico para a equipe de ML/CV, compartilhando conhecimento e melhores práticas.
    \item Desenvolvendo novas funcionalidades e otimizações para o sistema de classificação de carcaças.
    \item Projetei soluções de banco de dados de grafos para modelar relacionamentos complexos entre dados de rastreabilidade de produtos, permitindo análises avançadas e melhorando a transparência da cadeia de suprimentos.
    \item \textbf{Tecnologias principais:} Python (PyTorch, OpenCV, Django), Docker, Kubernetes, AWS, Neo4j
\end{itemize}

\vspace{12pt}
\textbf{Engenheiro Sênior de Inteligência Artificial} \hfill \textit{Novembro 2023 -- Dezembro 2023}
\begin{itemize}
    \item Liderei a expansão do sistema de visão computacional para múltiplas unidades industriais.
    \item Implementei melhorias no pipeline de inferência, reduzindo o tempo de processamento em 35\%.
    \item Desenvolvi novos módulos para análise de qualidade e rastreabilidade de produtos.
    \item Coordenei a equipe de ML/CV, estabelecendo processos e metodologias de desenvolvimento.
    \item Arquitetei soluções híbridas de dados usando cache distribuído para resultados de classificação em tempo real e bancos de dados de grafos para modelagem de relacionamentos complexos de produtos e redes da cadeia de suprimentos, resultando em 50\% de melhoria nas respostas de consultas para relatórios de rastreabilidade.
    \item Implantei soluções de cache para dados de sessão de estudantes, reduzindo carga do banco de dados e melhorando responsividade do sistema.
    \item \textbf{Tecnologias principais:} Python (PyTorch, OpenCV, Django), Docker, Kubernetes, AWS, CUDA, Neo4j
\end{itemize}

\vspace{12pt}
\textbf{Consultor de Visão Computacional} \hfill \textit{Novembro 2022 -- Outubro 2023}
\begin{itemize}
    \item Desenvolvi sistema de visão computacional usando PyTorch e OpenCV para classificação automática de carcaças bovinas, aumentando a eficiência de processamento em 25\%.
    \item Implementei técnicas avançadas de processamento de imagens com scikit-image e PIL para segmentação e análise de qualidade.
    \item Desenvolvi API REST com Django para integração do sistema de visão computacional com o sistema ERP existente.
    \item Otimizei o pipeline de inferência para processamento em tempo real usando CUDA e TensorRT.
    \item Implantei camada de cache para modelos de classificação e resultados frequentemente acessados, reduzindo carga do banco de dados em 60\% e melhorando tempos de resposta da API para requisitos de processamento em tempo real.
    \item \textbf{Tecnologias principais:} Python (PyTorch, OpenCV, Django, scikit-image, PIL), Docker, Kubernetes, AWS, CUDA, TensorRT, Redis, Neo4j
\end{itemize}

\vspace{24pt}
\small{\textbf{Hypeone - Curitiba, Brasil}}
\vspace{3pt}

\textbf{Consultor de Machine Learning} \hfill \textit{Janeiro 2023 -- Outubro 2023}
\begin{itemize}
    \item Atuei como consultor técnico para equipes de ML e desenvolvimento, fornecendo mentoria em projetos complexos.
    \item Otimizei arquitetura de microserviços com Spring Boot e Spring Cloud, melhorando a escalabilidade do sistema.
    \item Implementei melhorias nos serviços de processamento de transações financeiras usando Java e Spring Framework.
    \item Desenvolvi novas funcionalidades para o sistema web usando Angular e TypeScript.
    \item Projetei soluções backend abrangentes usando cache de alta performance para transações financeiras e bancos de dados de grafos para modelagem de redes complexas de relacionamento com clientes e padrões de detecção de fraudes, resultando em 45\% de melhoria na velocidade de processamento de transações.
    \item \textbf{Tecnologias principais:} Java (Spring Boot, Spring Cloud), Python (scikit-learn, FastAPI), Angular, Kafka, AWS, Redis, Neo4j
\end{itemize}

\vspace{12pt}
\textbf{Engenheiro Sênior de Software} \hfill \textit{Maio 2021 -- Dezembro 2022}
\begin{itemize}
    \item Desenvolvi sistema financeiro completo usando Java com Spring Boot, incluindo módulos de processamento de transações e análise de risco.
    \item Implementei arquitetura escalável de microserviços usando Spring Cloud, Service Discovery e API Gateway.
    \item Criei modelos de ML usando scikit-learn e PyCaret para automação de processos financeiros.
    \item Desenvolvi interfaces web modernas e responsivas usando Angular e TypeScript.
    \item Integrei sistemas usando Apache Kafka para processamento assíncrono e em tempo real.
    \item Arquitetei soluções de cache distribuído e gerenciamento de sessão, permitindo escalonamento horizontal e reduzindo consultas ao banco de dados em 70\% durante períodos de pico de transações.
    \item \textbf{Tecnologias principais:} Java (Spring Boot, Spring Cloud), Python (scikit-learn, FastAPI), Angular, Kafka, Docker, Redis, Neo4j
\end{itemize}

\vspace{24pt}
\small{\textbf{Universidade Federal do Paraná (VRI) - Curitiba, Brasil}}
\vspace{3pt}

\textbf{Pesquisador de Doutorado} \hfill \textit{Outubro 2016 -- Maio 2022}
\begin{itemize}
    \item Desenvolvi modelos de deep learning com PyTorch e TensorFlow para reconhecimento biométrico, focando em reconhecimento periocular.
    \item Implementei técnicas avançadas de processamento de imagens e visão computacional.
    \item Publiquei artigos em periódicos e conferências internacionais de alto impacto.
    \item Colaborei em projetos internacionais de pesquisa biométrica.
    \item \textbf{Tecnologias principais:} Python (PyTorch, TensorFlow, OpenCV), Docker, CUDA
\end{itemize}

\vspace{24pt}
\small{\textbf{Unicesumar - Maringá, Brasil}}
\vspace{3pt}

\textbf{Engenheiro Sênior de Software} \hfill \textit{Outubro 2016 -- Dezembro 2020}
\begin{itemize}
    \item Desenvolvi sistema de reconhecimento facial com PyTorch e OpenCV para monitoramento de exames online, reduzindo significativamente fraudes acadêmicas.
    \item Implementei modelos preditivos usando scikit-learn e Pandas para identificar estudantes em risco de evasão, reduzindo taxas em 87\%.
    \item Criei dashboards interativos com Plotly e Streamlit para visualização de métricas acadêmicas.
    \item Desenvolvi API REST com Django para integração do sistema de reconhecimento facial com o LMS.
    \item Projetei soluções de banco de dados de grafos para modelar padrões comportamentais complexos de estudantes e relacionamentos acadêmicos, permitindo análises preditivas que melhoraram estratégias de intervenção precoce e contribuíram para a redução de 87\% na evasão.
    \item Implantei soluções de cache para dados de sessão de estudantes, reduzindo carga do banco de dados e melhorando responsividade do sistema.
    \item \textbf{Tecnologias principais:} Python (PyTorch, scikit-learn, Django, OpenCV, Pandas, Plotly, Streamlit), Docker, AWS, Redis, Neo4j
\end{itemize}

\vspace{24pt}
\small{\textbf{Wasys - Curitiba, Brasil}}
\vspace{3pt}

\textbf{Cientista de Dados Sênior} \hfill \textit{Outubro 2018 -- Maio 2021}
\begin{itemize}
    \item Desenvolvi módulos completos de sistema ERP usando Java com JSF e Spring Boot, incluindo gestão financeira e controle de estoque.
    \item Implementei arquitetura de microserviços com Spring Boot, garantindo escalabilidade e manutenção eficiente.
    \item Desenvolvi sistema OCR usando TensorFlow e OpenCV para extração automática de dados de documentos financeiros, melhorando precisão em 18\%.
    \item Criei interfaces web responsivas e intuitivas usando JSF, PrimeFaces e Bootstrap.
    \item Implementei integrações com sistemas externos usando APIs REST e mensageria com Apache Kafka.
    \item Projetei soluções de banco de dados de grafos para modelar relacionamentos comerciais complexos no sistema ERP, permitindo análises avançadas em redes cliente-fornecedor e dependências de estoque, melhorando processos de tomada de decisão em 30\%.
    \item Implantei soluções de cache distribuído para dados ERP frequentemente acessados e gerenciamento de sessão, reduzindo carga do banco de dados e melhorando responsividade do sistema para usuários concorrentes em 55\%.
    \item \textbf{Tecnologias principais:} Java (Spring Boot, JSF, Hibernate), Python (TensorFlow, OpenCV), Docker, Kubernetes, Apache Kafka, Neo4j, Redis
\end{itemize}

\vspace{24pt}
\small{\textbf{Universidade Estadual de Maringá - Maringá, Brasil}}
\vspace{3pt}

\textbf{Pesquisador de Mestrado} \hfill \textit{Fevereiro 2014 -- Agosto 2016}
\begin{itemize}
    \item Desenvolvi aplicação de visão computacional para classificação automática de espécies de pássaros baseada em suas vocalizações.
    \item Implementei técnicas de processamento de sinais para conversão de áudio em espectrogramas usando bibliotecas especializadas.
    \item Apliquei descritores de textura avançados para extração de características de espectrogramas.
    \item Desenvolvi e otimizei modelos de classificação usando SVM (Support Vector Machines).
    \item \textbf{Tecnologias principais:} Python, MATLAB, OpenCV, scikit-learn, librosa
\end{itemize}

\vspace{24pt}
\small{\textbf{Seebot - Maringá, Paraná, Brasil}}
\vspace{3pt}

\textbf{Pesquisador Sênior de Biometria/Desenvolvedor Full Stack} \hfill \textit{Janeiro 2014 -- Dezembro 2016}
\begin{itemize}
    \item Desenvolvi algoritmos de visão computacional com Python e OpenCV para sistemas biométricos.
    \item Implementei sistemas embarcados usando Python em Raspberry Pi para captura e processamento de imagens em tempo real.
    \item Criei interfaces web com PHP e Java para visualização e gerenciamento de dados biométricos.
    \item Integrei sistemas de hardware (Arduino) com software para automação de captura biométrica.
    \item \textbf{Tecnologias principais:} Python (OpenCV, scikit-learn), Java, PHP, Arduino, Raspberry Pi
\end{itemize}

\vspace{24pt}
\small{\textbf{Elotech Informática e Sistemas Ltda - Maringá, Paraná, Brasil}}
\vspace{3pt}

\textbf{Desenvolvedor Júnior} \hfill \textit{Agosto 2013 -- Fevereiro 2014}
\begin{itemize}
    \item Desenvolvi soluções em Adobe Flex e Java para o portal de transparência.
    \item Colaborei com membros da equipe para garantir integração perfeita de novas funcionalidades.
    \item Implementei práticas de codificação eficientes para otimizar performance e experiência do usuário.
    \item \textbf{Tecnologias principais:} Adobe Flex, Java, JSP, JSF, JSTL, MySQL
\end{itemize}

\vspace{24pt}
\small{\textbf{Visãonet Telecom - Goioerê, Paraná, Brasil}}
\vspace{3pt}

\textbf{Desenvolvedor Júnior} \hfill \textit{Março 2012 -- Dezembro 2013}
\begin{itemize}
    \item Desenvolvi novas funcionalidades para sistemas web internos em JAVA e PHP.
    \item Mantive sites de e-commerce no framework MAGENTO.
    \item Colaborei com a equipe para garantir operações e atualizações suaves.
    \item Contribuí para o sucesso geral dos projetos de desenvolvimento web.
    \item \textbf{Tecnologias principais:} Java, PHP, MAGENTO, MySQL
\end{itemize}

\vspace{12pt}
\textbf{Analista de Suporte de Sistemas} \hfill \textit{Abril 2011 -- Março 2012}
\begin{itemize}
    \item Liderei a implementação de sistemas de automação comercial, otimizando operações e aumentando produtividade.
    \item Desenvolvi novas funcionalidades de software em resposta a solicitações de clientes.
    \item Colaborei com equipes para garantir integração perfeita de novas funcionalidades.
    \item Contribuí para a eficiência geral e funcionalidade do software.
    \item \textbf{Tecnologias principais:} Delphi, Firebird, Java, PHP, MySQL
\end{itemize}

\vspace{12pt}
\textbf{Operador de Call Center} \hfill \textit{Julho 2008 -- Abril 2010}
\begin{itemize}
    \item Forneci suporte ao cliente via telefone para resolver problemas de acesso à internet.
    \item Desenvolvi novas tecnologias e corrigi bugs em produtos da empresa para melhorar a experiência do cliente.
    \item Colaborei com membros da equipe para melhorar processos de atendimento ao cliente e eficiência.
    \item Implementei soluções inovadoras para otimizar operações de suporte ao cliente.
    \item \textbf{Tecnologias principais:} PHP, MySQL, Elastix
\end{itemize}

\vspace{24pt}
\small{\textbf{Papelaria Famsit - Goioerê, Paraná, Brasil}}
\vspace{3pt}

\textbf{Auxiliar Geral} \hfill \textit{Janeiro 2006 -- Junho 2008}
\begin{itemize}
    \item Gerenciei tarefas de limpeza, fotocópia e atendimento ao cliente.
    \item Garantei ambiente de trabalho limpo e organizado e forneci serviços eficientes.
    \item Implementei novos protocolos para melhorar eficiência e satisfação do cliente.
    \item Desenvolvi fortes habilidades multitarefas e atendimento ao cliente em ambiente de ritmo acelerado.
\end{itemize}

\vspace{12pt}
\section{Educação}

\cventry{2016--2022}{Doutorado em Ciência da Computação}{Universidade Federal do Paraná}{Curitiba, Paraná, Brasil}{}{}

\cventry{2014--2016}{Mestrado em Ciência da Computação}{Universidade Estadual de Maringá}{Maringá, Paraná, Brasil}{}{}

\cventry{2008--2011}{Tecnólogo em Sistemas para Internet}{Universidade Tecnológica Federal do Paraná}{Campo Mourão, Paraná, Brasil}{}{}

\vspace{12pt}
\section{Prêmios e Reconhecimentos}

\begin{itemize}

\item{\cventry{}{Primeiro lugar na Competição de Biometria (Região Periocular)}{IEEE WORLD CONGRESS ON COMPUTATIONAL INTELLIGENCE}{Glasgow, Reino Unido, 2020}{}{}}

\item{\cventry{}{Segundo lugar na Competição de Segmentação de Esclera}{IEEE INTERNATIONAL JOINT CONFERENCE ON BIOMETRICS}{Houston, EUA, 2020}{}{}}

\end{itemize}

\vspace{12pt}
\section{Publicações Selecionadas}

\begin{itemize}

\item Pati, S.; Baid, U.; Edwards, B. et al.(2022). Federated learning enables big data for rare cancer boundary detection. Nature communications, 13(1), 7346. \textbf{[305 citações]}

\item LUCIO, D. R. et al. (2020). COVID-19 detection in CT images with deep learning: A voting-based scheme and cross-datasets analysis. Informatics in Medicine Unlocked, 20, 100427. \textbf{[260 citações]}

\item LUCIO, D. R.; COSTA, Y. M. G. (2016). Combining visual and acoustic features for audio classification tasks. Pattern Recognition Letters, 88, 49-56. \textbf{[99 citações]}

\item ZANLORENSI, L. A.; LAROCA, R.; LUCIO, D. R.; SANTOS, L. R., BRITTO JR; A. S.; MENOTTI, D. (2022). A new periocular dataset collected by mobile devices in unconstrained scenarios. Scientific Reports, 12(1), 17989. \textbf{[26 citações]}

\item VITEK, MATEJ; DAS, ABHIJIT; LUCIO, DIEGO RAFAEL et al. (2022). Exploring Bias in Sclera Segmentation Models: A Group Evaluation Approach. IEEE Transactions on Information Forensics and Security, 18, 190-205. \textbf{[19 citações]}

\end{itemize}

\vspace{8pt}
Lista completa disponível no \href{https://scholar.google.com.br/citations?user=FS_momQAAAAJ&hl=en}{Google Scholar} .

\vspace{12pt}
\section{Apresentações de Trabalhos}

\begin{enumerate}
    \item LUCIO D. R. ; ZANLORENSI L.; BESSON V., COSTA Y. M. G., MENOTTI D. . Pupil Constrictions and Dilations Effects as Data Augmentation on an Iris Recognition CNN Approach. In: 23rd International Conference on Artificial Intelligence and Soft Computing, 2024.

    \item KIMURA G.Y. ; LUCIO D. R. ; BRITO A. S. ; MENOTTI D. . Simultaneous Iris and Periocular Region Detection Using Coarse Annotations. In: International Joint Conference on Computer Vision, Imaging and Computer Graphics Theory and Applications, 2020.

    \item LUCIO, D. R. ; LAROCA, R. ; SEVERO, E. ; BRITTO JR., A. S. ; MENOTTI, D. . Fully Convolutional Networks and Generative Adversarial Networks Applied to Sclera Segmentation. 2018.

\end{enumerate}

\end{document}

%% end of file `template.tex'.
